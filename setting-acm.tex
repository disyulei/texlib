% =============================================================
%         Setting for ACM 2017 format
%
%   Author      : Bei Yu
%   Last Update : 04/2018
% =============================================================

\settopmatter{printacmref=false}          % no ACM Reference Format
\fancyhead{}                              % no header

% ==== page margin settings
\geometry{twoside=true, head=13pt,
	paperwidth=8.5in, paperheight=11in,
	includeheadfoot, columnsep=2pc,
	top=57pt, bottom=73pt, inner=54pt, outer=54pt,
	marginparwidth=2pc,heightrounded
}%

\iffalse
% === shrink page num
\usepackage{titlesec}
\titlespacing\section{0pt}{3pt plus 1pt minus 1pt}{0pt plus 1pt minus 1pt}
\titlespacing\subsection{0pt}{3pt plus 1pt minus 1pt}{0pt plus 1pt minus 1pt}
\titlespacing\subsubsection{0pt}{3pt plus 1pt minus 1pt}{2pt plus 1pt minus 1pt}

\setlength{\textfloatsep}{7pt plus 1pt minus 1pt}         % set space between float and text
\setlength{\floatsep}{10pt plus 1pt minus 1pt}             % set space between two floats
\setlength{\intextsep}{4pt plus 1pt minus 1pt}             % set space between text and float
\setlength{\columnsep}{16pt}                               % set space between columns
% ==== reduce space around equations
\setlength{\belowdisplayskip}{4pt} \setlength{\belowdisplayshortskip}{4pt}
\setlength{\abovedisplayskip}{4pt} \setlength{\abovedisplayshortskip}{4pt}
\fi
%\intextsep             = 0pt plus 1pt minus 7pt           % set space above and below the texts
%\dbltextfloatsep       = 0pt plus 1pt minus 7pt
%\dblfloatsep           = 0pt plus 1pt minus 7pt
%\abovedisplayskip      = 0pt plus 1pt minus 7pt           % set space around equations
%\belowdisplayskip      = 0pt plus 1pt minus 7pt
%\abovedisplayshortskip = 0pt plus 1pt minus 7pt
%\belowdisplayshortskip = 0pt plus 1pt minus 7pt

% ==== Packages
\usepackage{lipsum}
\usepackage{graphicx}
\usepackage{footnote}
\newcommand{\bmmax}{0}    % fix problems associated with mathrsfs
\newcommand{\hmmax}{0}
\usepackage{amsmath}
%\usepackage{amssymb}
\usepackage{mathtools}
\usepackage{mathrsfs}     % mathscr command
\usepackage{comment}
\usepackage[subrefformat=parens,labelformat=parens]{subfig}
\captionsetup[subfigure]{labelformat=simple}               % avoid "double brackets" in sub-figure caption
\renewcommand\thesubfigure{(\alph{subfigure})}             % "Fig.~1b"-->"Fig.1(b)"
\usepackage{bm}
\usepackage{multirow}
\usepackage{threeparttable,booktabs}
\usepackage{blkarray}
\usepackage{tikz}
\usetikzlibrary{positioning,calc,fit,decorations.pathmorphing,shapes.geometric,shapes.gates.logic.US,calc}
\usepackage{balance}
\usepackage{courier}                                       % courier font, used in \texttt
\usepackage{cleveref}                                      % smart citation
%\Crefformat{figure}{Fig.~#2#1#3}                           % "Fig.", instead of "Figure"
%\Crefname{subfigure}{Fig.}{Figs.}
%\Crefname{figure}{Fig.}{Figs.}
\usepackage[mathcal]{eucal}
%\usepackage[noend]{algorithm2e}
\usepackage[]{algpseudocode}                               % algorithm package
%\usepackage[noend]{algpseudocode}
\algrenewcommand\textproc{\texttt}
\makeatletter\let\float@addtolists\relax\makeatother
\usepackage{algorithm}
\renewcommand{\algorithmicrequire}{\textbf{Input:} }       % Use Input in the format of Algorithm
\renewcommand{\algorithmicensure} {\textbf{Output:}}       % Use Output in the format of Algorithm
\usepackage{filecontents}                                  % support to pgfplots
\usepackage{pgfplots}
\usepackage{pgfplotstable}
\usepgfplotslibrary{groupplots}
%\usepackage{pgf-pie}
\pgfplotsset{compat=newest}
\usepackage[figuresright]{rotating}
\iffalse   % minted setting
\IfFileExists{minted_configuration.tex}{\input{minted_configuration.tex}}{\usepackage[outputdir=../]{minted}}
\setminted{linenos=true, escapeinside=@@}
\fi


% ==== Local new commands
\newcommand{\calH}{\mathcal{H}}
\newcommand{\calN}{\mathcal{N}}
\newcommand{\calO}{\mathcal{O}}
\newcommand{\calP}{\mathcal{P}}
\newcommand{\calV}{\mathcal{V}}
\newcommand{\norm}[1]{\left\lVert#1\right\rVert}
\newcommand{\m}[1]{\boldsymbol{#1}}
\renewcommand{\vec}[1]{\boldsymbol{#1}}
\newcommand{\todo}[1]{\textcolor{red}{[TODO: #1]}}
\newcommand{\revise}[1]{\textcolor{blue}{#1}}
\DeclareMathOperator*{\argmin}{argmin}
\DeclareMathOperator*{\argmax}{argmax}

\theoremstyle{plain}
\newtheorem{mytheorem}{\textbf{Theorem}}
\newtheorem{mylemma}{\textbf{Lemma}}
\newtheorem{myclaim}{\textbf{Claim}}
\newtheorem{myproperty}{\textbf{Property}}

\theoremstyle{definition}
\newtheorem{mydefinition}{\textbf{Definition}}
\newtheorem{myproblem}{\textbf{Problem}}

\algrenewcommand\textproc{\texttt}
\newcommand{\tabincell}[2]{
    \begin{tabular}{@{}#1@{}}#2\end{tabular}
}

% ==== spacing control on caption
\usepackage[skip=1pt]{caption}            % set space between figure and caption
\setlength{\belowcaptionskip}{-1.0mm}
\captionsetup[table]{aboveskip=5pt}       % reduce space around table caption
\captionsetup[table]{belowskip=2pt}

% ==== local color definitions
\definecolor{CUHKorange}{RGB}{244,106,18} %F47012
\definecolor{CUHKblue}{RGB}{0,111,190}    %006FBE
\definecolor{CUHKgreen}{RGB}{0,127,128}   %007F80
\definecolor{CUHKred}{RGB}{228,46,36}     %E42E24
\definecolor{CUHKyellow}{RGB}{198,148,34} %C69422
\definecolor{CUHKdark}{RGB}{114,44,114}   %722C72
\definecolor{CUHKmiddle}{RGB}{144,44,144} %902C90

% === beamer style block
\usepackage{tcolorbox}
\tcbuselibrary{skins,breakable}
\newenvironment{myblock}[1]{%
    \tcolorbox[beamer,%
    noparskip,breakable,
    colback=white,colframe=CUHKmiddle,%
    colbacklower=white,%
    title=#1]}%
    {\endtcolorbox}

