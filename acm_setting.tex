% =============================================================
%            settign for ACM format paper
%
%   Author      : Bei Yu
%   Last Update : 11/2015
% =============================================================

% =====  including package
\usepackage[subrefformat=parens,farskip=0pt,justification=centering]{subfig}
\usepackage{graphicx}
\usepackage{threeparttable}
\usepackage{ntheorem}
\usepackage{color}
\usepackage{multirow}
\usepackage{amsmath}
\usepackage{comment}
\usepackage{bm}                      % bold math font
\usepackage{etoolbox}                % commands \newtoggle, \toggletrue, \iftoggle
\usepackage{url}
\usepackage{nth}                     % nth command
\usepackage{cite}
\usepackage{enumitem}
\usepackage{hyperref}
\usepackage{balance} 
\setitemize{noitemsep,topsep=0pt,parsep=0pt,partopsep=0pt}

% =====  algorithm setting
%\usepackage[noend]{algpseudocode}
\usepackage[]{algpseudocode}
\usepackage{algorithm}
\renewcommand{\algorithmicrequire}{\textbf{Input:}}  % Use Input in the format of Algorithm
\renewcommand{\algorithmicensure}{\textbf{Output:}}  % Use Output in the format of Algorithm


%theorem
\theoremstyle{plain}
\theoremheaderfont{\normalfont\bfseries}\theorembodyfont{\slshape}
\theoremsymbol{\ensuremath{\diamondsuit}} \theoremseparator{:}
\newtheorem{mytheorem}{Theorem}


% problem
\theoremstyle{plain}
\theoremsymbol{\ensuremath{\clubsuit}}
\theoremseparator{.}
\newtheorem{myproblem}{Problem}

% application
\theoremstyle{plain}
\theoremsymbol{\ensuremath{\clubsuit}}
\theoremseparator{.}
\newtheorem{myapplication}{Application}

% metric
\theoremstyle{plain}
\theoremsymbol{\ensuremath{\clubsuit}}
\theoremseparator{.}
\newtheorem{mymetric}{Metric}


%definition
\theoremstyle{plain}
\theoremsymbol{\ensuremath{\clubsuit}}
\theoremseparator{.}
\newtheorem{mydefinition}{Definition}

%lemma
%\theoremstyle{changebreak}
\theoremsymbol{\ensuremath{\heartsuit}}
%\theoremindent0.5cm
%\theoremnumbering{greek}
\newtheorem{mylemma}{Lemma}

%claim
%\theoremstyle{changebreak}
\theoremsymbol{\ensuremath{\heartsuit}}
%\theoremindent0.5cm
%\theoremnumbering{greek}
\newtheorem{myclaim}{Claim}

%proof
\theoremheaderfont{\sc}\theorembodyfont{\upshape}
\theoremstyle{nonumberplain} \theoremseparator{}
\theoremsymbol{\rule{1ex}{1ex}}
\newtheorem{myproof}{Proof}

%example
\theoremstyle{change}
\theorembodyfont{\upshape}
\theoremsymbol{\ensuremath{\ast}}
\theoremseparator{}
\newtheorem{myxample}{Example}

%%%%%%%%%%%%%%%%%%%%%%%%%%%% Setting to control figure placement
% These determine the rules used to place floating objects like figures
% They are only guides, but read the manual to see the effect of each.
%\renewcommand{\topfraction}{1.9}
%\renewcommand{\bottomfraction}{1.9}
%\renewcommand{\textfraction}{.1}
%


\makeatletter
\def\bstctlcite{\@ifnextchar[{\@bstctlcite}{\@bstctlcite[@auxout]}}
\def\@bstctlcite[#1]#2{\@bsphack
  \@for\@citeb:=#2\do{%
    \edef\@citeb{\expandafter\@firstofone\@citeb}%
    \if@filesw\immediate\write\csname #1\endcsname{\string\citation{\@citeb}}\fi}%
  \@esphack}
\makeatother

\makeatletter
 \let\@copyrightspace\relax
\makeatother


% ==== Log
%
%  11/2015: use 'balance', balance two-column mode paper bottom
%  11/2015: use 'hyperref'
%  11/2015: use 'subfloat', instead of 'subfigure'
%  12/2014: nth package
%  11/2014: add newtheorem "myapplication"
%  11/2014: add package "etoolbox"
%  10/2014: add package "authblk" & "bm"
%  10/2014: add "bstctlcite"
%  06/2014: forbid package "cite", check why!
%


