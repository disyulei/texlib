% =============================================================
%         Setting for CV papers
%
%   Author      : Bei Yu
%   Last Update : 11/2021
% =============================================================

% === comment out for ICCV, but used for CVPR
\makeatletter
\@namedef{ver@everyshi.sty}{}
\makeatother

\usepackage{times}
\usepackage{epsfig}
\usepackage{graphicx}
\usepackage{amsmath,amsfonts,amssymb,amsthm}
\usepackage[mathcal]{eucal}                                % charp math font
%\usepackage{subfig}                                       % conflict with subcaption so removed from CVPR22
\captionsetup[subfigure]{labelformat=simple}               % avoid "double brackets" in sub-figure caption
\renewcommand\thesubfigure{(\alph{subfigure})}             % "Fig.~1b"-->"Fig.1(b)"
%\usepackage[colorlinks]{hyperref}
\usepackage[pagebackref=true,breaklinks=true,letterpaper=true,colorlinks,bookmarks=false]{hyperref}
\hypersetup{
    pageanchor=false,
    linkcolor=blue,
    anchorcolor=blue,
    citecolor=blue,
    urlcolor=blue,
    plainpages=false,
}
\usepackage{cleveref}
%\usepackage{floatrow}
\usepackage{caption}
\usepackage{algpseudocode}                            % new algorithm package
\usepackage{algorithm}
\renewcommand{\algorithmicrequire}{\textbf{Input:} }  % Use Input in the format of Algorithm
\renewcommand{\algorithmicensure} {\textbf{Output:}}  % Use Output in the format of Algorithm
\usepackage{tikz}
\usepackage{pgfplots}
\usepackage{pgfplotstable}
\pgfplotsset{compat=newest}
\usepackage{filecontents}
\usepackage{booktabs}                                 % professional-quality tables
\usepackage{tabularx}
\usepackage{multirow}
\usepackage{color}
\usepackage[flushleft]{threeparttable}
\usepackage{bm}
\usepackage{textcomp}                                 % for degree symbol
\usepackage{enumitem}
\usepackage{setspace}                                 % setstretch command

\iffalse
\setlength{\textfloatsep}{10pt plus 1pt minus 1pt}    % set space between float and text
\setlength{\floatsep}{10pt plus 1pt minus 1pt}        % set space between two floats
\setlength{\intextsep}{4pt plus 1pt minus 1pt}        % set space between text and float
\setlength{\columnsep}{16pt}                          % set space between columns
% ==== reduce space around equations
\setlength{\belowdisplayskip}{2pt} \setlength{\belowdisplayshortskip}{2pt}
\setlength{\abovedisplayskip}{2pt} \setlength{\abovedisplayshortskip}{2pt}
\fi


% ==== Local new commands
\newcommand{\calH}{\mathcal{H}}
\newcommand{\calN}{\mathcal{N}}
\newcommand{\calO}{\mathcal{O}}
\newcommand{\calP}{\mathcal{P}}
\newcommand{\calQ}{\mathcal{Q}}
\newcommand{\calV}{\mathcal{V}}
\newcommand{\norm}[1]{\left\lVert#1\right\rVert}
\newcommand{\tensor}[1]{\boldsymbol{\mathcal{#1}} }    % tensor 
\renewcommand{\vec}[1]{\boldsymbol{#1}}                % re-define vec command
\DeclareMathOperator*{\argmin}{argmin}
\DeclareMathOperator*{\argmax}{argmax}

% ==== Theorem Definitions
\newtheorem{myproblem}{\textbf{Problem}}
\newtheorem{myassumption}{\textbf{Assumption}}
\newtheorem{mydefinition}{\textbf{Definition}}
\newtheorem{mytheorem}{\textbf{Theorem}}
\newtheorem{mycorollary}{\textbf{Corollary}}
\newtheorem{mylemma}{\textbf{Lemma}}
\newtheorem{myclaim}{\textbf{Claim}}
\newtheorem{myapplication}{\textbf{Application}}
\newtheorem{myexample}{\textbf{Example}}
\newtheorem{myconjecture}{Conjecture}
\newtheorem{myprop}{Property}



