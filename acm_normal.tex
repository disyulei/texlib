% =============================================================
%         Setting for ACM format paper (normal mode)
%
%   Author      : Bei Yu
%   Last Update : 11/2016
% =============================================================


% =============================================================
%         particular setting for normal mode
% =============================================================
\textfloatsep          = 10pt plus 1pt minus 2pt           % set space between float and text
\usepackage[labelfont=bf]{caption}                         % set caption font size
\usepackage[subrefformat=parens]{subfig}
\captionsetup[table]{aboveskip=0pt}                        % reduce space around table caption
\captionsetup[table]{belowskip=6pt}
\usepackage{enumitem}                                      % item spacing


% =============================================================
%                 Common setting
% =============================================================
\DeclareCaptionType{copyrightbox}
\usepackage{graphicx}
\usepackage{threeparttable}
\usepackage{ntheorem}
\usepackage{color}
\usepackage{colortbl}
\usepackage{multirow}
\usepackage{amsmath}
\usepackage{comment}
\usepackage{bm}                                            % bold math font
\usepackage{etoolbox}                                      % \newtoggle, \toggletrue, \iftoggle
\usepackage{url}
\usepackage{nth}
\usepackage{cite}
\usepackage{balance}
\usepackage{courier}                                       % courier font, used in \texttt 
\usepackage{booktabs}                                      % \toprule \bottomrule
\usepackage{listings}                                      % typeset source code listings
\lstset{basicstyle=\ttfamily,breaklines=true}
\usepackage[]{algpseudocode}                               % algorithm package
%\usepackage[noend]{algpseudocode}
\algrenewcommand\textproc{\texttt}
\makeatletter\let\c@float@type\relax\makeatother
\makeatletter\let\float@addtolists\relax\makeatother
\usepackage{algorithm}
\renewcommand{\algorithmicrequire}{\textbf{Input:} }       % Use Input in the format of Algorithm
\renewcommand{\algorithmicensure} {\textbf{Output:}}       % Use Output in the format of Algorithm
\usepackage[bookmarks=false]{hyperref}                     %
\usepackage[hyphenbreaks]{breakurl}                        % break long url
\hypersetup{
    colorlinks = true,
    citecolor  = blue,
    linkcolor  = blue,
	urlcolor   = blue,
}
\usepackage{filecontents}                                  % support to pgfplots
\usepackage{pgfplots}
\usepackage{pgfplotstable}
\pgfplotsset{compat=newest}


\newcommand{\tabincell}[2]{                                % newline in table cell
	\begin{tabular}{@{}#1@{}}#2\end{tabular}
}

%theorem
\theoremstyle{plain}
\theoremheaderfont{\normalfont\bfseries}
\theorembodyfont{\slshape}
\theoremsymbol{\ensuremath{\diamondsuit}} \theoremseparator{:}
\newtheorem{mytheorem}{Theorem}

% problem
\theoremstyle{plain}
\theoremheaderfont{\normalfont\bfseries}
\theorembodyfont{\slshape}
\theoremsymbol{\ensuremath{\clubsuit}}
\theoremseparator{.}
\newtheorem{myproblem}{Problem}

%lemma
\theoremstyle{plain}
\theoremsymbol{\ensuremath{\heartsuit}}
\theoremsymbol{\ensuremath{\diamondsuit}} \theoremseparator{:}
\newtheorem{mylemma}{Lemma}

% corollary
\theoremstyle{plain}
\theoremsymbol{\ensuremath{\clubsuit}}
\theoremsymbol{\ensuremath{\diamondsuit}} \theoremseparator{:}
\newtheorem{mycorollary}{Corollary}

%definition
\theoremstyle{definition}
\theoremheaderfont{\normalfont\bfseries}
\theorembodyfont{\normalfont}
\theoremsymbol{\ensuremath{\clubsuit}}
\theoremseparator{.}
\newtheorem{mydefinition}{Definition}

% application
\theoremstyle{definition}
\theoremheaderfont{\normalfont\bfseries}
\theorembodyfont{\normalfont}
\theoremsymbol{\ensuremath{\clubsuit}}
\theoremseparator{.}
\newtheorem{myapplication}{Application}

%claim
\theoremstyle{definition}
\theoremsymbol{\ensuremath{\heartsuit}}
\theoremseparator{.}
\newtheorem{myclaim}{Claim}

%example
\theoremstyle{definition}
\theoremsymbol{\ensuremath{\heartsuit}}
\theoremseparator{.}
\newtheorem{myexample}{Example}

%Property
\theoremstyle{definition}
\theoremsymbol{\ensuremath{\heartsuit}}
\theoremseparator{.}
\newtheorem{myprop}{Property}

%proof
\theoremheaderfont{\sc}\theorembodyfont{\upshape}
\theoremstyle{nonumberplain} \theoremseparator{}
\theoremsymbol{\rule{1ex}{1ex}}
\newtheorem{myproof}{Proof}

\makeatletter
\def\bstctlcite{\@ifnextchar[{\@bstctlcite}{\@bstctlcite[@auxout]}}
\def\@bstctlcite[#1]#2{\@bsphack
  \@for\@citeb:=#2\do{%
    \edef\@citeb{\expandafter\@firstofone\@citeb}%
    \if@filesw\immediate\write\csname #1\endcsname{\string\citation{\@citeb}}\fi}%
  \@esphack}
\makeatother

% =============================================================
%          Definitions to support latexdiff
% =============================================================
%DIF PREAMBLE EXTENSION ADDED BY LATEXDIFF
%DIF UNDERLINE PREAMBLE %DIF PREAMBLE
\RequirePackage[normalem]{ulem} %DIF PREAMBLE
\RequirePackage{color}\definecolor{RED}{rgb}{1,0,0}\definecolor{BLUE}{rgb}{0,0,1} %DIF PREAMBLE
\providecommand{\DIFadd}[1]{{\protect\color{blue}{#1}}}                           %DIF PREAMBLE
\providecommand{\DIFdel}[1]{{\protect\color{red}\sout{#1}}}                       %DIF PREAMBLE
%DIF SAFE PREAMBLE %DIF PREAMBLE
\providecommand{\DIFaddbegin}{}                                                   %DIF PREAMBLE
\providecommand{\DIFaddend}{}                                                     %DIF PREAMBLE
\providecommand{\DIFdelbegin}{}                                                   %DIF PREAMBLE
\providecommand{\DIFdelend}{}                                                     %DIF PREAMBLE
%DIF FLOATSAFE PREAMBLE %DIF PREAMBLE
\providecommand{\DIFaddFL}[1]{\DIFadd{#1}}                                        %DIF PREAMBLE
\providecommand{\DIFdelFL}[1]{\DIFdel{#1}}                                        %DIF PREAMBLE
\providecommand{\DIFaddbeginFL}{}                                                 %DIF PREAMBLE
\providecommand{\DIFaddendFL}{}                                                   %DIF PREAMBLE
\providecommand{\DIFdelbeginFL}{}                                                 %DIF PREAMBLE
\providecommand{\DIFdelendFL}{}                                                   %DIF PREAMBLE
%DIF END PREAMBLE EXTENSION ADDED BY LATEXDIFF


% ==== Log
%
%  11/2016: package booktabs; theorem/definition formats
%  10/2016: pgfplots support
%  08/2016: break long url
%  07/2016: captionsetup
%  07/2016: correct format for myexample, colortbl
%  04/2016: set font of algorithmic function name to \texttt{}
%  04/2016: remove bookmarks in hyperref
%  03/2016: definitions supporting latexdiff
%  02/2016: courier, listings
%  02/2016: tabincell
%  11/2015: add page size setting
%  11/2015: use 'subfig' to replace 'subfigure'
%  11/2015: use 'balance', balance two-column mode paper bottom
%  11/2015: use 'hyperref'
%  12/2014: nth package
%  11/2014: add newtheorem "myapplication"
%  11/2014: add package "etoolbox"
%  10/2014: add package "authblk" & "bm"
%  10/2014: add "bstctlcite"
%  06/2014: forbid package "cite", check why!
%

