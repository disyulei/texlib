% =============================================================
%            setting for IEEE format paper
%
%   Author      : Bei Yu
%   Last Update : 11/2015
% =============================================================


%\usepackage[noend]{algpseudocode}
\usepackage{algpseudocode}                                 % new algorithm package
\usepackage{algorithm}
\usepackage{graphicx}                                      % include pdf figures
\usepackage{amsmath}
\usepackage{amssymb}
\usepackage{amsfonts}
\usepackage{booktabs}
\usepackage{multirow}
\usepackage[subrefformat=parens,farskip=0pt,justification=centering]{subfig}
\usepackage{color}
\usepackage{cite}                                          % more citations in one bracket
\usepackage{comment}                                       % use comment
\usepackage{soul}                                          % use highlight command \hl{}
\soulregister\cite7
\soulregister\ref7
\soulregister\pageref7
\usepackage{amsthm}
\usepackage{etoolbox}                                      % commands \newtoggle, \toggletrue, \iftoggle
\usepackage{url}
\usepackage{nth}                                           % nth command
\usepackage{bm}                                            % bm command
\setlength{\textfloatsep}{10.0pt plus 1.0pt minus 2.0pt}   % set space between float and text


% =============================================================
%                    page size setting
% =============================================================
\paperwidth   = 8.5in                                      % US Letter
\paperheight  = 11.0in
%\paperwidth  = 8.26in                                     % A4
%\paperheight = 11.69in
\usepackage[top=0.8in,bottom=0.8in,left=0.6in,right=0.6in]{geometry}


% =============================================================
%                   Theorem Definitions
% =============================================================
\newtheorem{myproblem}{\textbf{Problem}}
\newtheorem{mydefinition}{\textbf{Definition}}
\newtheorem{mytheorem}{\textbf{Theorem}}
\newtheorem{mylemma}{\textbf{Lemma}}
\newtheorem{myclaim}{\textbf{Claim}}
\newtheorem{myapplication}{\textbf{Application}}
\newtheorem{myconjecture}{Conjecture}


% long line in algorithm
% e.g.: \Statex[4] ...;
\makeatletter
\let\OldStatex\Statex
\renewcommand{\Statex}[1][3]{%
  \setlength\@tempdima{\algorithmicindent}%
  \OldStatex\hskip\dimexpr#1\@tempdima\relax
}
\makeatother


% ==== Logs:
%
%  11/2015: use 'hyperref'
%  11/2015: use 'subfloat', instead of 'subfigure'
%  06/2015: clean IEEEtranBSTCTL commands
%  11/2014: \solregister, to support highlight to mybox, definition, and citations
%  10/2014: add package "authblk" & "bm"
%  09/2014: "problem" => "myproblem"
%  07/2014: introduce \Statex[] for lone algorithmic statement
%  05/2014: new package {soul} to use highlight command \hl{}
%

